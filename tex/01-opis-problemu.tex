\subsection{Treść zadania}
\label{subsec:polecenie}
Należy zaprojektować system sterowania manipulatorem o~sześciu stopniach swobody, wyposażony w~chwytak dwustanowy oraz kamerę RGB-D (Kinect). Na taśmociągu poruszają się różnokolorowe sześciany o~wymiarach 4cm~$\pm$~1cm. Zadaniem robota jest pobieranie żółtych sześcianów poruszających się na czarnym taśmociągu i~układanie ich na palecie o~wymiarach 100cm~x~100cm. Sześciany mają być ustawione na~palecie w~konfiguracji 20x20. 

Szybkość ruchu taśmociągu jest stała i~wynosi $\num{0,1}\frac{m}{s}$ - taśmociąg nie jest sterowany przez projektowany system. Pozycja taśmociągu oraz kamery względem podstawy robota jest znana (określa je projektant systemu). Sześciany spadają pojedynczo na początek taśmociągu co 40~sekund. Ich położenie początkowe i~orientacja są losowe. Szerokość taśmociągu wynosi $\num{0.3}$ m, a~jego długość $\num{1,2}$ m. System rozpoczyna pracę po otrzymaniu komendy \texttt{START}, a~kończy ją gdy paleta się zapełni. Komendy \texttt{START} wydawane są przez zdalnego agenta, którego definiować nie potrzeba. Wymiana palet jest zadaniem innych urządzeń, które nie są pod kontrolą projektowanego systemu.

Stosując formalizm przedstawiony na wykładzie należy:
\begin{itemize}
    \item określić strukturę systemu w~kategoriach agentów,
    \item dla każdego agenta należy zdefiniować podsystem sterowania, efektory i~receptory wirtualne,
    \item dla tych podsystemów określić:
    \begin{itemize}
        \item automat skończony sterujący ich pracą,
        \item zachowania,
        \item warunki początkowe i~końcowe zachowań,
        \item funkcje przejścia w~postaci matematycznej i~DFD,
        \item zawartość pamięci wewnętrznej oraz buforów wejściowych i~wyjściowych,
        \item krok dyskretyzacji dla każdego podsystemu.
    \end{itemize}
\end{itemize}
