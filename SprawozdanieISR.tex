\documentclass{article}
\pdfpagewidth=8.5in
\pdfpageheight=11in

\usepackage{ISRreport}
\usepackage{times}
\usepackage{url}
\usepackage{xcolor}
\usepackage{polski}
\usepackage[polish]{babel}
\usepackage[utf8]{inputenc}
\usepackage[T1]{fontenc}
\usepackage[utf8]{luainputenc}
\usepackage[hidelinks]{hyperref}
\usepackage[utf8]{inputenc}

%%% fix for \lll
\let\babellll\lll
\let\lll\relax

\usepackage{amssymb}
\usepackage{amsmath}
\usepackage{caption}
\usepackage{indentfirst}
\usepackage{graphicx}
\usepackage{amsmath}
\usepackage{siunitx}
\usepackage{booktabs}
\usepackage{subfig}
\usepackage{pgfplots}
\usepackage{paracol}
\usepackage{gensymb}
\usepackage{float}
\pagestyle{plain}
	
\title{Inteligentne Systemy Robotyczne \\ Zadanie magazynowania sześcianów}

\author{
Jakub Sikora
\affiliations
numer albumu: 283418 \\
\emails
jakub.sikora2.stud@pw.edu.pl
}

\newcommand{\todo}[1]{\textcolor{red}{\textbf{TO DO:} #1}}

\begin{document}
\maketitle

\section{Opis problemu}
\label{sec:opis-problemu}
\subsection{Treść zadania}
\label{subsec:polecenie}
Należy zaprojektować system sterowania manipulatorem o~sześciu stopniach swobody, wyposażony w~chwytak dwustanowy oraz kamerę RGB-D (Kinect). Na taśmociągu poruszają się różnokolorowe sześciany o~wymiarach 4cm~$\pm$~1cm. Zadaniem robota jest pobieranie żółtych sześcianów poruszających się na czarnym taśmociągu i~układanie ich na palecie o~wymiarach 100cm~x~100cm. Sześciany mają być ustawione na~palecie w~konfiguracji 20x20. 

Szybkość ruchu taśmociągu jest stała i~wynosi $\num{0,1}\frac{m}{s}$ - taśmociąg nie jest sterowany przez projektowany system. Pozycja taśmociągu oraz kamery względem podstawy robota jest znana (określa je projektant systemu). Sześciany spadają pojedynczo na początek taśmociągu co 40~sekund. Ich położenie początkowe i~orientacja są losowe. Szerokość taśmociągu wynosi $\num{0.3}$ m, a~jego długość $\num{1,2}$ m. System rozpoczyna pracę po otrzymaniu komendy \texttt{START}, a~kończy ją gdy paleta się zapełni. Komendy \texttt{START} wydawane są przez zdalnego agenta, którego definiować nie potrzeba. Wymiana palet jest zadaniem innych urządzeń, które nie są pod kontrolą projektowanego systemu.

Stosując formalizm przedstawiony na wykładzie należy:
\begin{itemize}
    \item określić strukturę systemu w~kategoriach agentów,
    \item dla każdego agenta należy zdefiniować podsystem sterowania, efektory i~receptory wirtualne,
    \item dla tych podsystemów określić:
    \begin{itemize}
        \item automat skończony sterujący ich pracą,
        \item zachowania,
        \item warunki początkowe i~końcowe zachowań,
        \item funkcje przejścia w~postaci matematycznej i~DFD,
        \item zawartość pamięci wewnętrznej oraz buforów wejściowych i~wyjściowych,
        \item krok dyskretyzacji dla każdego podsystemu.
    \end{itemize}
\end{itemize}


\section{Struktura systemu}
\label{sec:struktura}
\subsection{Konfiguracja rzeczywista}
\label{subsec:konfiguracja-urzadzen}

\begin{figure}
    \centering
    \includegraphics[width=\columnwidth]{figures/ISR-system-overview.pdf}
    \caption{Środowisko robocze projektowanego systemu}
    \label{fig:srodowisko-robocze}
\end{figure}

Projektowany system składa się z:\begin{itemize}
    \item sześciostopniowego manipulatora,
    \item chwytaka dwustanowego,
    \item kamery RGB-D,
    \item ruchomego taśmociągu,
    \item palety do odstawiania sześcianów.
\end{itemize}

Pozycja tych komponentów została przedstawiona na rysunku~\ref{fig:srodowisko-robocze}. Wszystkie transformacje pomiędzy poszczególnymi układami współrzędnych komponentów są znane.

\subsection{Agentowa struktura systemu}
\label{subsec:agentowa-struktura}

W~systemie można wyróżnić dwóch agentów: agenta $a_{1}$ wykonującego zadanie i~sterującego ramieniem, chwytakiem i~kamerą oraz agenta $a_{2}$.  Diagram komunikacji pomiędzy agentami został przedstawiony na rysunku~\ref{fig:agenty-system}. Zgodnie z~poleceniem, definicja agenta $a_{2}$ zostanie pominięta.

\begin{figure}[b]
    \centering
    \includegraphics[width=\columnwidth]{figures/ISR-agents.pdf}
    \caption{Dekompozycja systemu na agenty}
    \label{fig:agenty-system}
\end{figure}

\begin{figure}[t]
    \centering
    \includegraphics[width=\columnwidth]{figures/ISR-agent-decomposition.pdf}
    \caption{Dekompozycja agenta $a_{1}$ na wirtualne i~rzeczywiste receptory i~efektory}
    \label{fig:dekompozycja-agent-1}
\end{figure}

Agent $a_{1}$ steruje dwoma efektorami: sześciostopniowym manipulatorem oraz dwustanowym chwytakiem, a~także odbiera dane z~jednego receptora: kamery RGB-D. Agenta $a_{1}$ zdekomponowano zgodnie z~formalizmem przedstawionym na wykładzie, na podsystem sterowania, wirtualne receptory i~efektory oraz rzeczywiste receptory i~efektory. Na rysunku~\ref{fig:dekompozycja-agent-1}, przedstawiono podział na poszczególne podsystemy, wraz z~odpowiadającymi buforami komunikacyjnymi.

\section{Podsystem sterowania}
\label{sec:cs}
\subsection{Struktura podsystemu}
\label{subsec:cs-struktura}

\begin{figure}[ht]
    \leftskip-2em
    \includegraphics[width=1.15\columnwidth]{figures/ISR-cs-model.pdf}
    \caption{Struktura ogólna podsystemu sterowania}
    \label{fig:model-cs}
\end{figure}

Na rysunku~\ref{fig:model-cs} przedstawiono widok podsystemu sterowania w~systemie. Jego rolą w systemie jest nadzorowanie pracy pozostałych podsystemów oraz komunikacja z~innymi agentami. W~podsystemie celowo pominięto bufor do nadawania komunikatów do innych agentów, ponieważ zgodnie z~poleceniem, agent nie nadaje żadnych wiadomości do innych agentów. Krok dyskretyzacji podsystemu sterowania jest dostosowany do działania kamery i~wynosi ${}^{c}T = \frac{1}{30}$s.

\subsubsection{Pamięć wewnętrzna}
W~pamięci wewnętrznej podsystemu przechowywane są informacje na temat jego działania.
\begin{equation}
    {}^{c}c_{1,1} = [M, i, j, \Theta_{\mathrm{plan}}]
\end{equation}

\begin{itemize}
    \item $M$ -- macierz zer i~jedynek rozmiaru 20x20 opisująca zajętość palety,
    \item $i,j$ -- współrzędne rozważanego miejsca na palecie, numerowane od 1 do 20, 0 oznacza niewyznaczoną pozycję,
    \item $\Theta_{\mathrm{plan}}$ -- zmienna przechowująca zaplanowaną pozycję chwytu/odłożenia sześcianu.
\end{itemize}

Oprócz zmiennych, w~pamięci wewnętrznej przechowywane są stałe pozycje, służące do przygotowania systemu do chwytania i~odkładania.

\begin{itemize}
    \item $\Theta_{\mathrm{grip}}$ -- pozycja startowa do chwytu sześcianu, 
    \item $\Theta_{\mathrm{drop}}$ -- pozycja startowa do odkładania sześcianu.
\end{itemize}
    
\subsubsection{Bufory komunikacyjne}
Podsystem komunikuje się z~wirtualnymi efektorami, wirtualnymi receptorami oraz innymi agentami za pomocą następujących buforów komunikacyjnych.
\begin{itemize}
    \item ${}^{T}_{x}c_{2,1} = m \in \{ \emptyset, START \}$ -- komunikat od agenta $a_{2}$,
    
    \item ${}^{r}_{y}r_{1,1} = \varphi \in \{b, p\}$ -- wybór trybu pracy wirtualnego receptora kamery,
    \item ${}^{r}_{x}r_{1,1} = \Theta_{\mathrm{d}}$ -- znaleziona pozycja sześcianu/miejsca w~zależności od wybranego trybu,

    \item ${}^{e}_{y}e_{1,1} = \Theta_{\mathrm{zad}}$ -- zadana pozycja ramienia,
    \item ${}^{e}_{x}e_{1,1} = \Theta$ -- aktualna pozycja ramienia,

    \item ${}^{e}_{y}e_{1,2} = \xi_{\mathrm{zad}} \in \{o, c\}$ -- zadany stan chwytaka,
    \item ${}^{e}_{x}e_{1,2} = \xi \in \{o, c\}$ -- aktualny stan chwytaka.
\end{itemize}

\subsubsection{Funkcje pomocnicze}
Do poprawnego działania podsystemu, wymagana jest implementacja kilku funkcji pomocniczych, których konkretna definicja pozostaje poza zakresem projektu.

\begin{itemize}
    \item \texttt{zeros()} -- stwórz macierz 20x20 wypełnioną zerami,
    \item \texttt{ $isValid(\Theta)$ } -- sprawdza czy $\Theta$ jest poprawną pozycją znajdującą się w~przestrzeni operacyjnej robota,
    \item \texttt{makePlan($\Theta, \Theta_{\mathrm{d}}$)} -- na podstawie aktualnego położenia $\Theta$, wykrytego położenia $\Theta_{\mathrm{d}}$ oraz znanej prędkości taśmociągu, określa położenie w~którym robot złapie sześcian, 
    \item \texttt{findPlace($\Theta_{\mathrm{d}}, M$)} --na podstawie położeń wykrytych potencjalnych miejsc $\Theta_{\mathrm{d}}$ oraz znanej zajętości palety $M$, określa położenie w~którym robot odłoży sześcian,
    \item \texttt{isFull($M$)} -- sprawdza czy podana macierz zajętości palety jest wypełniona (czy należy zakończyć pracę).
\end{itemize}

\subsection{Automat sterujący}
\label{subsec:cs-automat}

\begin{figure}[ht]
    \leftskip-5em
    \includegraphics[ width=1.4\columnwidth]{figures/ISR-cs-behaviours.pdf}
    \caption{Automat zachowań podsystemu sterowania}
    \label{fig:zachowania-cs}
\end{figure}

Na rysunku~\ref{fig:zachowania-cs} przedstawiony został automat sterujący podsystemem sterowania. Z~każdym stanem $\{ {}^{c}S_{1,1}^0, \hdots, {}^{c}S_{1,1}^7 \}$ zostało skojarzone zachowanie $\{ {}^{c}\mathcal{B}_{1,1,0}, \hdots, {}^{c}\mathcal{B}_{1,1,7} \}$. Każdemu z~zachowań została nadana nazwa, w~celu uproszczenia dalszego opisu:
\begin{itemize}
    \item ${}^{c}\mathcal{B}_{1,1,0}$ - idle,
    \item ${}^{c}\mathcal{B}_{1,1,1}$ - pre-grip,
    \item ${}^{c}\mathcal{B}_{1,1,2}$ - detect-block,
    \item ${}^{c}\mathcal{B}_{1,1,3}$ - do-plan,
    \item ${}^{c}\mathcal{B}_{1,1,4}$ - grip,
    \item ${}^{c}\mathcal{B}_{1,1,5}$ - pre-store,
    \item ${}^{c}\mathcal{B}_{1,1,6}$ - detect-place,
    \item ${}^{c}\mathcal{B}_{1,1,7}$ - drop.
\end{itemize}

\subsection{Zachowanie idle}
\label{subsec:cs-idle}

Zachowanie ${}^{c}\mathcal{B}_{1,1,0}$, zwane dalej zachowaniem \textbf{idle}, jest pierwszym zachowaniem, które przejmuje kontrolę nad systemem. Jego celem jest oczekiwanie na komunikat \texttt{START} od agenta $a_{2}$.

\subsubsection{Funkcja przejścia}
\begin{equation}
{}^{c_{1,1}, c_{1,1}}f_{1,1,0} \triangleq {}^{c}c_{1,1} = [\text{\texttt{zeros()}}, 0, 0, \emptyset]  
\end{equation}

\begin{figure}[ht]
    \leftskip1.5em
    \includegraphics[width=\columnwidth]{figures/ISR-cs-fp-idle.pdf}
    \caption{Zdekomponowana funkcja przejścia zachowania \textbf{idle} w~postaci DFD}
    \label{fig:cs-fp-idle}
\end{figure}

\subsubsection{Warunki początkowe}
\begin{equation}
    {}^{c}f^{\sigma}_{1,1,7,0} \triangleq \Xi = o \land \text{\texttt{isFull($M$)}}
\end{equation}

\subsubsection{Warunki końcowe}
\begin{equation}
    {}^{c}f^{\tau}_{1,1,0} \triangleq {}^{T}_{x}c_{2,1} = \text{\texttt{START}}
\end{equation}

%%%%%%%%%%%%%%%%%%%%%%%%%%%%%%%%%%%%%%%%%%%%%%%%%%%%%%%%%%%%%%%%%%%%%%%%%%%%%%%%%%%%%

\subsection{Zachowanie pre-grip}
\label{subsec:cs-pre-grip}

Rolą zachowania ${}^{c}\mathcal{B}_{1,1,1}$ (\textbf{pre-grip}), jest przygotowanie systemu do złapania sześcianu. W~tym celu, ramię robota zostanie ustawione w~pozycji $\Theta_{\mathrm{grip}}$ w~której to kamera znajdzie się nad taśmociągiem, aby móc obserwować pojawiające się na nim sześciany. Dodatkowo, w~tym zachowaniu, podsystem sterowania wysyła komendy do chwytaka dwustanowego, aby otworzył swój uchwyt.

\subsubsection{Funkcja przejścia}
\begin{equation}
    \begin{gathered}
        {}^{c_{1,1}, e_{1,1}}f_{1,1,1} \triangleq {}^{e}_{y}c_{1,1} = \Theta_{\mathrm{grip}},
        \\
        {}^{c_{1,1}, e_{1,2}}f_{1,1,1} \triangleq {}^{e}_{y}c_{1,2} = o
    \end{gathered}
\end{equation}
    
\begin{figure}[ht]
    \leftskip2.5em
    \includegraphics[width=\columnwidth]{figures/ISR-cs-fp-pre-grip.pdf}
    \caption{Zdekomponowana funkcja przejścia zachowania \textbf{pre-grip} w~postaci DFD}
    \label{fig:cs-fp-pre-grip}
\end{figure}

\subsubsection{Warunki początkowe}
\begin{equation}
    \begin{gathered}
        {}^{c}f^{\sigma}_{1,1,0,1} \triangleq {}^{T}_{x}c_{2,1} = START\\
        {}^{c}f^{\sigma}_{1,1,7,1} \triangleq \Xi = o \land \neg \text{\texttt{isFull($M$)}}
    \end{gathered}
\end{equation}

\subsubsection{Warunki końcowe}
\begin{equation}
    {}^{c}f^{\tau}_{1,1,1} \triangleq \Theta = \Theta_{\mathrm{grip}} \land \Xi = o
\end{equation}

%%%%%%%%%%%%%%%%%%%%%%%%%%%%%%%%%%%%%%%%%%%%%%%%%%%%%%%%%%%%%%%%%%%%%%%%%%%%%%%%%%%%%

\subsection{Zachowanie detect-block}
\label{subsec:cs-detect-block}
W~trakcie stanu aktywności zachowania ${}^{c}\mathcal{B}_{1,1,2}$ (\textbf{detect-block}), robot obserwuje taśmociąg, na którym pojawiają się kolorowe sześciany. Funkcja przejścia oczekuje aż wirtualny receptor, który przetwarza obraz z~kamery, zwróci poprawną pozycję wykrytego, żółtego sześcianu i~na tej podstawie oraz znajomości prędkości taśmociągu, określa pozycję w~której dojdzie do chwytu.

\subsubsection{Funkcja przejścia}
\begin{equation}
    \begin{gathered}
      {}^{c_{1,1}, r_{1,1}}f_{1,1,2} \triangleq {}^{r}_{y}c_{1,1} = b\\        
      {}^{c_{1,1}, c_{1,1}}f_{1,1,2} \triangleq \\ \Theta_{\mathrm{plan}} =
                   \begin{cases}
       			    \text{\texttt{makePlan($\Theta, \Theta_{\mathrm{d}}$)}}, & \text{\texttt{isValid($\Theta_{\mathrm{d}}$)}}\\
                       \emptyset, & \text{w p.p.}
       		    \end{cases}
    \end{gathered}
\end{equation}

\begin{figure}[ht]
    \leftskip1.5em
    \includegraphics[width=\columnwidth]{figures/ISR-cs-fp-detect-block.pdf}
    \caption{Zdekomponowana funkcja przejścia zachowania \textbf{detect-block} w~postaci DFD}
    \label{fig:cs-fp-detect-block}
\end{figure}

\subsubsection{Warunki początkowe}
\begin{equation}
    {}^{c}f^{\sigma}_{1,1,1,2} \triangleq {}^{c}f^{\tau}_{1,1,1} = True
\end{equation}

\subsubsection{Warunki końcowe}
\begin{equation}
    {}^{c}f^{\tau}_{1,1,2} \triangleq \text{\texttt{isValid($\Theta_{\mathrm{d}}$)}}
\end{equation}


%%%%%%%%%%%%%%%%%%%%%%%%%%%%%%%%%%%%%%%%%%%%%%%%%%%%%%%%%%%%%%%%%%%%%%%%%%%%%%%%%%%%%

\subsection{Zachowanie do-plan}
\label{subsec:cs-do-plan}
Zachowanie ${}^{c}\mathcal{B}_{1,1,3}$ (\textbf{do-plan}), realizuje wcześniej wygenerowany plan $\Theta_{plan}$, który został zapisany w~pamięci wewnętrznej podsystemu.

\subsubsection{Funkcja przejścia}
\begin{equation}
    {}^{c_{1,1}, e_{1,1}}f_{1,1,3} \triangleq {}^{e}_{y}c_{1,1} = \Theta_{\mathrm{plan}}
\end{equation}

\begin{figure}[ht]
    \leftskip1.5em
    \includegraphics[width=\columnwidth]{figures/ISR-cs-fp-do-plan.pdf}
    \caption{Zdekomponowana funkcja przejścia zachowania \textbf{do-plan} w~postaci DFD}
    \label{fig:cs-fp-do-plan}
\end{figure}

\subsubsection{Warunki początkowe}
\begin{equation}
    \begin{gathered}
        {}^{c}f^{\sigma}_{1,1,2,3} \triangleq \text{\texttt{isValid($\Theta_{\mathrm{d}}$)}} = True \\
        {}^{c}f^{\sigma}_{1,1,6,3} \triangleq \text{\texttt{isValid($\Theta_{\mathrm{d}}$)}} = True        
    \end{gathered}
\end{equation}

\subsubsection{Warunki końcowe}
\begin{equation}
    {}^{c}f^{\tau}_{1,1,3} \triangleq \Theta = \Theta_{\mathrm{plan}}
\end{equation}


%%%%%%%%%%%%%%%%%%%%%%%%%%%%%%%%%%%%%%%%%%%%%%%%%%%%%%%%%%%%%%%%%%%%%%%%%%%%%%%%%%%%%

\subsection{Zachowanie grip}
\label{subsec:cs-grip}
W~trakcie zachowania ${}^{c}\mathcal{B}_{1,1,4}$, chwytak robota łapie wykryty klocek.R
System oczekuje na moment w~którym klocek znajdzie się w~przewidywanej pozycji, trzymając chwytak w~gotowości. Chwytak zaciska się w~momencie gdy wykryta pozycja klocka znajdzie się pomiędzy jego palcami. Zachowanie kończy się w~momencie uzyskania informacji o~uzyskaniu maksymalnego zamknięcia chwytaka.

\subsubsection{Funkcja przejścia}
\begin{equation}
    {}^{c_{1,1}, e_{1,2}}f_{1,1,4} \triangleq {}^{e}_{y}c_{1,2} = \begin{cases}
        c, & \Theta_{\mathrm{plan}} = \Theta_{\mathrm{d}}\\
        o, & \Theta_{\mathrm{plan}} \neq \Theta_{\mathrm{d}}
    \end{cases}
\end{equation}

\begin{figure}[ht]
    \leftskip1.5em
    \includegraphics[width=\columnwidth]{figures/ISR-cs-fp-grip.pdf}
    \caption{Zdekomponowana funkcja przejścia zachowania \textbf{grip} w~postaci DFD}
    \label{fig:cs-fp-grip}
\end{figure}

\subsubsection{Warunki początkowe}
\begin{equation}
    {}^{c}f^{\sigma}_{1,1,3,4} \triangleq {}^{c}f^{\tau}_{1,1,3} = True \land \Xi = o
\end{equation}

\subsubsection{Warunki końcowe}
\begin{equation}
    {}^{c}f^{\tau}_{1,1,4} \triangleq \Xi = c
\end{equation}

%%%%%%%%%%%%%%%%%%%%%%%%%%%%%%%%%%%%%%%%%%%%%%%%%%%%%%%%%%%%%%%%%%%%%%%%%%%%%%%%%%%%%

\subsection{Zachowanie pre-drop}
\label{subsec:cs-pre-drop}
Zachowanie ${}^{c}\mathcal{B}_{1,1,5}$ (\textbf{pre-drop}) jest bliźniaczo podobne do zachowania~\textbf{pre-grip}, jednak różni się pozycją docelową. Dodatkowo, w~trakcie jego aktywności, robot cały czas musi trzymać sześcian w~chwytaku. Pozycja robota $\Theta_{\mathrm{drop}}$ została tak przygotowana, aby robot mógł bez problemu obserwować całą paletę do składowania sześcianów.

\subsubsection{Funkcja przejścia}
\begin{equation}
    {}^{c_{1,1}, e_{1,1}}f_{1,1,5} \triangleq {}^{e}_{y}c_{1,1} = \Theta_{\mathrm{drop}}
\end{equation}

\begin{figure}[ht]
    \leftskip1.5em
    \includegraphics[width=\columnwidth]{figures/ISR-cs-fp-pre-drop.pdf}
    \caption{Zdekomponowana funkcja przejścia zachowania \textbf{pre-drop} w~postaci DFD}
    \label{fig:cs-fp-pre-drop}
\end{figure}

\subsubsection{Warunki początkowe}
\begin{equation}
    {}^{c}f^{\sigma}_{1,1,4,5} \triangleq \Xi = c
\end{equation}

\subsubsection{Warunki końcowe}
\begin{equation}
    {}^{c}f^{\tau}_{1,1,5} \triangleq \Theta = \Theta_{\mathrm{drop}}
\end{equation}

%%%%%%%%%%%%%%%%%%%%%%%%%%%%%%%%%%%%%%%%%%%%%%%%%%%%%%%%%%%%%%%%%%%%%%%%%%%%%%%%%%%%%

\subsection{Zachowanie detect-place}
\label{subsec:cs-detect-place}
Zachowanie ${}^{c}\mathcal{B}_{1,1,6}$ (\textbf{detect-place}) jest odpowiednikiem zachowania \textbf{detect-block}, z~tą różnicą że w~trakcie jego aktywności wykrywane są docelowe miejsca na odłożenie. Na tej podstawie wybierane jest konkretne miejsce na palecie (współrzędne $i,j$) oraz plan operacji $\Theta_{plan}$.

\subsubsection{Funkcja przejścia}
\begin{equation}
    \begin{gathered}
        {}^{c_{1,1}, r_{1,1}}f_{1,1,6} \triangleq {}^{r}_{y}c_{1,1} = p\\
        {}^{c_{1,1}, c_{1,1}}f_{1,1,6} \triangleq \\ [\Theta_{\mathrm{plan}}, i, j] =
            \begin{cases}
			    \text{\texttt{findPlace($\Theta, \Theta_{\mathrm{d}}, M$)}}, & \text{\texttt{isValid($\Theta_{\mathrm{d}}$)}}\\
                [\emptyset, 0, 0], & \text{w p.p.}
		    \end{cases}
    \end{gathered}
\end{equation}

\begin{figure}[ht]
    \leftskip1.5em
    \includegraphics[width=\columnwidth]{figures/ISR-cs-fp-detect-place.pdf}
    \caption{Zdekomponowana funkcja przejścia zachowania \textbf{detect-place} w~postaci DFD}
    \label{fig:cs-fp-detect-place}
\end{figure}

\subsubsection{Warunki początkowe}
\begin{equation}
    {}^{c}f^{\sigma}_{1,1,5,6} \triangleq {}^{c}f^{\tau}_{1,1,5} = True
\end{equation}

\subsubsection{Warunki końcowe}
\begin{equation}
    {}^{c}f^{\tau}_{1,1,6} \triangleq \text{\texttt{isValid($\Theta_{\mathrm{d}}$)}}
\end{equation}

%%%%%%%%%%%%%%%%%%%%%%%%%%%%%%%%%%%%%%%%%%%%%%%%%%%%%%%%%%%%%%%%%%%%%%%%%%%%%%%%%%%%%

\subsection{Zachowanie drop}
\label{subsec:cs-drop}
Ostatnim rozważanym zachowaniem w~podsystemie sterowania jest ${}^{c}\mathcal{B}_{1,1,6}$ (\textbf{drop}), podczas którego robot uwalnia sześcian ze swojego uchwytu, zostawiając go w~zaplanowanym miejscu. Robot odznacza w~pamięci, które komórki palety zostały zajęte, dzięki czemu możliwe jest stwierdzenie czy zadanie zostało zakończone oraz w~przeciwnym przypadku, umożliwia dalsze planowanie czynności.

\subsubsection{Funkcja przejścia}
\begin{equation}
    \begin{gathered}
        {}^{c_{1,1}, c_{1,1}}f_{1,1,7} \triangleq M[i,j] = 1 \\
        {}^{c_{1,1}, e_{1,2}}f_{1,1,7} \triangleq {}^{e}_{y}c_{1,2} = o
    \end{gathered}
\end{equation}

\begin{figure}[ht]
    \leftskip1em
    \includegraphics[width=\columnwidth]{figures/ISR-cs-fp-drop.pdf}
    \caption{Zdekomponowana funkcja przejścia zachowania \textbf{drop} w~postaci DFD}
    \label{fig:cs-fp-drop}
\end{figure}

\subsubsection{Warunki początkowe}
\begin{equation}
    {}^{c}f^{\sigma}_{1,1,3,7} \triangleq {}^{c}f^{\tau}_{1,1,3} = True \land \Xi = c
\end{equation}

\subsubsection{Warunki końcowe}
\begin{equation}
    {}^{c}f^{\tau}_{1,1,7} \triangleq \Xi = o
\end{equation}


\section{Wirtualny efektor manipulatora}
\label{sec:ve-manip}
\begin{figure}
    \centering
    \includegraphics[width=0.75\columnwidth]{figures/ISR-ve-manip-model.pdf}
    \label{fig:model-vr-camera}
    \caption{Struktura ogólna wirtualnego efektora manipulatora sześciostopniowego}
\end{figure}

\begin{figure}
    \centering
    \includegraphics[width=\columnwidth]{figures/ISR-ve-manip-behaviours.pdf}
    \label{fig:zachowania-ve-manip}
    \caption{Automat zachowań wirtualnego efektora manipulatora sześciostopniowego}
\end{figure}

Efektor wirtualny manipulatora:
\begin{itemize}
    \item bufor wejściowy od podsystemu sterowania: pozycja pożądana,
    \item bufor wyjściowy do podsystemu sterowania: pozycja aktualna,
    \item bufor wejściowy od rzeczywistego efektora: aktualne położenie wałów silnika,
    \item bufor wyjściowy do rzeczywistego efektora: pożądana pozycja wałów silnika.
\end{itemize}


Zachowania:
\begin{itemize}
    \item ${}^{e}\mathcal{B}_{1,1,0}$ - idle,
    \item ${}^{e}\mathcal{B}_{1,1,1}$ - move.
\end{itemize}

\section{Wirtualny efektor chwytaka}
\label{sec:ve-gripper}
\subsection{Struktura wirtualnego efektora}
\label{subsec:ve-gripper-struktura}

\begin{figure}[ht]
    \centering
    \includegraphics[width=0.75\columnwidth]{figures/ISR-ve-gripper-model.pdf}
    \caption{Struktura ogólna wirtualnego efektora chwytaka dwustanowego}
    \label{fig:model-ve-gripper}
\end{figure}

Na rysunku~\ref{fig:model-ve-gripper} przedstawiono widok wirtualnego efektora chwytaka dwustanowego w~projektowanym systemie. Jego rolą jest nadzorowanie pracą rzeczywistego efektora sterującego chwytakiem. Do poprawnej pracy podsystemu wymagane są wszystkie cztery bufory komunikacyjne, jednak nie potrzebuje on do działania wewnętrznej pamięci. Podobnie jak efektor manipulatora, krok dyskretyzacji został ustawiony na ${}^{e}T = \frac{1}{30}s$.

\subsubsection{Bufory komunikacyjne}
\begin{itemize}
    \item ${}^{c}_{x}e_{1,2} = \xi_{\mathrm{zad}} \in \{o, c\}$ - zadany stan chwytaka,
    \item ${}^{c}_{y}e_{1,2} = \xi \in \{o, c\}$ - aktualny stan chwytaka przesyłany do podsystemu sterowania,
    \item ${}^{E}_{x}e_{1,2} = \Xi \in \{o, c\}$ - aktualny stan chwytaka, aktualizowany po pełnym rozwarciu chwytaka lub po maksymalnym domknięciu,
    \item ${}^{E}_{y}e_{1,2} = \Xi_{\mathrm{zad}} \in \{o, c\}$ - aktualnie realizowany stan chwytaka.
\end{itemize}

\subsection{Automat sterujący}
Prostota działania chwytaka implikuje jeden stan normalnej pracy urządzenia, z~którym skojarzone zostało zachowanie ${}^{e}\mathcal{B}_{1,2,0}$ (\textbf{work}). Zachowanie \textbf{work} jest jedynym zachowaniem podsystemu, dlatego nie ma potrzeby rozważania warunków początkowych oraz końcowych. 

\subsubsection{Funkcja przejścia}
\begin{equation}
    \begin{gathered}
        {}^{e_{1,2}, E_{1,2}}f_{1,2,0} \triangleq {}^{E}_{y}e_{1,2} = \xi,\\
        {}^{e_{1,2}, c_{1,1}}f_{1,2,0} \triangleq {}^{c}_{y}e_{1,2} = \Xi
    \end{gathered}
\end{equation}

\begin{figure}
    \centering
    \includegraphics[width=\columnwidth]{figures/ISR-ve-gripper-fp-work.pdf}
    \label{fig:ve-gripper-fp-work}
    \caption{Zdekomponowana funkcja przejścia zachowania \textbf{work} w~postaci DFD}
\end{figure}


%%%%%%%%%%%%%%%%%%%%%%%%%%%%%%%%%%%%%%%%%%%%%%%%%%%%%%%%%%%%%%%%%%%%%%%%%%%%%%%%%

\section{Wirtualny receptor kamery RGB-D}
\label{sec:vr-camera}
\subsection{Struktura wirtualnego receptora}
\label{subsec:vr-camera-struktura}

\begin{figure}[ht]
    \centering
    \includegraphics[width=0.75\columnwidth]{figures/ISR-vr-camera-model.pdf}
    \caption{Struktura ogólna wirtualnego receptora kamery RGB-D}
    \label{fig:model-vr-camera}
\end{figure}

Na rysunku~\ref{fig:model-vr-camera} przedstawiono strukturę wirtualnego receptora. Jego rolą jest odbiór surowych danych pomiarowych z~rzeczywistego receptora oraz ich obróbka i~agregacja przed wysłaniem ich do podsystemu sterowania. Do poprawnej pracy podsystemu, wymagane są trzy bufory komunikacyjne: dwa do komunikacji z~podsystemem sterowania oraz jeden do odbioru danych z~kamery. Krok dyskretyzacji wirtualnego receptora jest wąskim gardłem systemu. Założono że receptor jest w~stanie działać z krokiem ${}^{r}T = \frac{1}{30}$s, czyli takim z~jakim działa wirtualny receptor. Założenie to jest wykonalne z~racji niskiej rozdzielczości kamer tego typu.

\subsubsection{Bufory komunikacyjne}
\begin{itemize}
    \item ${}^{c}_{x}r_{1,1} = \varphi \in \{b, p\}$ - tryb detekcji klocków/miejsc,
    \item ${}^{c}_{y}r_{1,1} = \Theta_{d}$ - pozycja wykrywanego klocka/miejsca,
    \item ${}^{R}_{x}r_{1,1} = \Lambda$ - chmura punktów z~kamery RGB-D.
\end{itemize}

\subsubsection{Funkcje pomocnicze}
Do poprawnego działania receptora, wymagana jest implementacja kilku funkcji pomocniczych, których konkretna definicja wymaga dokładnych parametrów omawianej kamery oraz budowy palety. Ich nagłówki zostały ograniczone do minimum, tak aby przedstawić ideę przetwarzania w~podsystemie.

\begin{itemize}
    \item \texttt{detect($\Lambda$)} - wykryj żółty sześciany
    \item \texttt{getTF($c$)} - zwróć pozycję podanego sześcianu,
    
    \item \texttt{find($\Lambda$)} - wykryj możliwe położenia do odstawienia,
    \item \texttt{getFinal($s$)} - zwróć pozycję docelowych położeń do odstawienia sześcianu, wraz z~ich koordynatami na palecie.
\end{itemize}

\subsection{Automat sterujący}
\begin{figure}[ht]
    \leftskip-3.5em
    \includegraphics[width=1.3\columnwidth]{figures/ISR-vr-camera-behaviours.pdf}
    \caption{Automat zachowań wirtualnego receptora kamery RGB-D}
    \label{fig:zachowania-vr-camera}
\end{figure}


Na rysunku~\ref{fig:zachowania-vr-camera} przedstawiony został automat sterujący wirtualnym receptorem kamery. Z~każdym z~dwóch stanów $ {}^{r}S_{1,1}^0,  {}^{r}S_{1,1}^1$ skojarzone zostało zachowanie ${}^{r}\mathcal{B}_{1,1,0}$ (\textbf{blocks}) oraz ${}^{r}\mathcal{B}_{1,1,1}$ (\textbf{places}). Każde z~zachowań określa tryb przetwarzania danych z~kamery. Wybór zachowań następuje poprzez wysłanie odpowiedniej wartości na bufor wejściowy. 

Dla podanego podsystemu, nie ma potrzeby badać warunków rozłączności, ponieważ dla każdego stanu w~automacie, istnieje tylko jeden stan do którego można przejść. Z~racji że dla każdego $i$, ${}^{r}f^{\sigma}_{1,1,j,i} = \neg {}^{r}f^{\sigma}_{1,1,j,i}$, udowadnia to spełnienie warunku kompletność stanów następnych.

\subsection{Zachowanie blocks}
\label{subsec:vr-camera-blocks}
Zachowanie ${}^{r}\mathcal{B}_{1,1,0}$ określa tryb wykrywania żółtych sześcianów. Gdy to zachowanie jest aktywne, na bufor ${}^{c}_{y}r_{1,1} = \Theta_{d}$ wysyłana jest pozycja wykrytego klocka lub wartość pusta. 

\subsubsection{Funkcja przejścia}
\begin{equation}
    {}^{r_{1,1}, c_{1,1}}f_{1,1,0} \triangleq {}^{c}_{y}e_{1,1} = \text{\texttt{getTF(detect(}}\Lambda\text{\texttt{))}}    
\end{equation}

\begin{figure}[ht]
    \leftskip-1.5em
    \includegraphics[width=\columnwidth]{figures/ISR-vr-camera-fp-blocks.pdf}
    \label{fig:vr-camera-fp-blocks}
    \caption{Zdekomponowana funkcja przejścia zachowania \textbf{blocks} w~postaci DFD}
\end{figure}

\subsubsection{Warunki początkowe}
\begin{equation}
    {}^{r}f^{\sigma}_{1,1,1,0} \triangleq \varphi = b
\end{equation}

\subsubsection{Warunki końcowe}
\begin{equation}
    {}^{r}f^{\tau}_{1,1,0} \triangleq \varphi \neq b
\end{equation}

\subsection{Zachowanie places}
\label{subsec:vr-camera-places}
Zachowanie ${}^{r}\mathcal{B}_{1,1,1}$ określa tryb wykrywania miejsc na palecie. Gdy to zachowanie jest aktywne, na bufor ${}^{c}_{y}r_{1,1} = \Theta_{d}$ wysyłana jest lista dostępnych pozycji wraz z~ich koordynatami na palecie lub wartość pusta. 

\subsubsection{Funkcja przejścia}
\begin{equation}
    {}^{r_{1,1}, c_{1,1}}f_{1,1,1} \triangleq {}^{c}_{y}e_{1,1} = \text{\texttt{getFinal(find(}}\Lambda\text{\texttt{))}}  
\end{equation}

\begin{figure}[ht]
    \leftskip-1.5em
    \includegraphics[width=\columnwidth]{figures/ISR-vr-camera-fp-places.pdf}
    \label{fig:vr-camera-fp-places}
    \caption{Zdekomponowana funkcja przejścia zachowania \textbf{places} w~postaci DFD}
\end{figure}

\subsubsection{Warunki początkowe}
\begin{equation}
    {}^{r}f^{\sigma}_{1,1,1,0} \triangleq \varphi = p
\end{equation}

\subsubsection{Warunki końcowe}
\begin{equation}
    {}^{r}f^{\tau}_{1,1,0} \triangleq \varphi \neq p
\end{equation}


\end{document}